% Chapter 10: Budget and sustainability

\chapter{Budget and Sustainability}

\label{chapter10}

\section{Budget}

Once the project is planned in time and its technologies are drafted, we can calculate the project’s budget.
Skyscanner is not economically transparent, even for the employees. So, this whole calculation will be approximated.
\\\\
First of all, we have to take into account the employees, which is only one and in Intern position, and also count the taxes.
\\\\
Apart from that, all the hardware material and software licenses. AWS costs will continue but since it is going to be used during the development for testing it is counted in the total budget.

\begin{table}[H]
\centering
\begin{tabular}{|l|l|l|l|l|}
\hline
\textbf{Concept}           & \textbf{Price per unit} & \textbf{Units} & \textbf{Amortization} & \textbf{Total £}   \\ \hline
\textbf{Salary}            & 28 (taxes included)     & 575 h          &                       & 16100              \\ \hline
\textbf{MacBook Pro}       & 1700                    & 1              & life cycle 8 years    & 210                \\ \hline
\textbf{JetBrains License} & 230                     & 1              & 1 year per developer  & 130                \\ \hline
\textbf{AWS S3}            & 0.023                   & 50 TB          &                       & 1150               \\ \hline
\textbf{AWS}               & EC2                     & 575 h          &                       & 213.325            \\ \hline
\textbf{Screen}            & 200                     & 2              & life cycle 4 years    & 100                \\ \hline
\textbf{Office}            & Uknown                  & 1              &                       & Uknown             \\ \hline
\textbf{TOTAL}             &                         &                &                       & \textbf{17903.325} \\ \hline
\end{tabular}
\caption{Budget calculation}
\label{budget-calculation}
\end{table}

\section{Sustainability}

\subsection{Economical}

In economic terms, this project is initially unsustainable. It uses resources from Skyscanner for a comparison that might be useful in the future for other projects, improving some services or advertisement.
\\\\
But, if this product is sell to providers, Skyscanner can take a lot of profit from them. It is a very valuable application for providers, since they could compare airlines offer with actual user demand. Letting them improve their flights distribution and make more money.

\subsection{Social}

The \thesistitle is not directly involving society, but, as explained before, if providers have access to the comparison, flights will improve in terms of traveler experience. Travelers will have accurate routes depending on what they really want.
\\\\
For example, imagine that X carrier have several flights from BCN to ORY, Paris, and a few from BCN to FCO, Rome. The \thesistitle shows that the demand, compared with the offer is bigger in Rome than in Paris. Then, X airline could schedule more flights to FCO instead of ORY.

\subsection{Environment}

The environmental impact of the \thesistitle is directly related with the social impact.
\\\\
Right now, some airlines may have half full flights. This means that the airplane is not taking its most advantage of the fuel. It could be carrying more people.
\\\\
If carriers know where flights are really needed those flight will be full of people, which means that the fuel a flight uses is profited at its most.
\\\\
Otherwise, if an offer is under requested, the flight is not giving all the profit it could.
In other words, fuel per person will decrease.

\subsection{Sustainability matrix}

In order to understand the general impact of the project, the following general rating and evaluation is provided:
\\\\
The economical impact will be rated in 7/10. It could be a 10/10 it is sell to providers. Air companies could pay a lot of money for the \thesistitle because of the information it provides.
\\\\
The social impact will get a 4/10. It does nothing good nor bad to the society, only if the application is sell to providers and they use it properly, it could make some good to the people. In the other hand, the software will not be free, it will be property of \company.
\\\\
The environmental impact is a 4/10 as well. The environmental impact could be good if the application is sell to providers and they use it properly, but for now will not be sell to anyone. It gets a 4 because it will be using Amazon Web Service, and those machines are powered mainly by non-renewable energy\cite{click_clean}.
\\\\

\begin{table}[H]
\centering
\begin{tabular}{|l|l|l|}
\hline
Economical & Social & Environmental \\ \hline
7/10       & 4/10   & 4/10          \\ \hline
\multicolumn{3}{|l|}{15/30}         \\ \hline
\end{tabular}
\caption{Sustainability matrix}
\label{sustainability-matrix}
\end{table}

