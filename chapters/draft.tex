Simple approximation for one year:
\\
62 thousand routes per week, around 20\% of those routes are new or changed from previous version, this is $62000+12400\times52weeks=706800$ records per year.
\\
In the other hand we have 4 million users every day, supposing the 75\% of them do one simple query (origin, destination and date): $4000000\times365=1460000000$ records.
\\
With a simple data model such as origin (Integer, 4 bytes), destination (Integer, 4 bytes) and date (Float, 4 bytes), each record could take 12 Bytes.
\\
$(4000000\times365+706800)\times12 Bytes = 17528481600 Bytes = \textbf{17.5 TB}$

%----------------------------------------------------------------
%----------------------------------------------------------------

% move to another chapter and section
\subsubsection*{Timetable SFN Service} \label{timetable_sfn_service}

The \textit{timetable SFN} endpoint returns details for time tabled Single Flight Number itineraries series. Note that SFNs are not ticket-able, so they do not include itineraries which cannot be bought on their own, neither the price nor restrictions.

% move to another chapter and section
\subsubsection*{Timetable Pipeline} \label{timetable_pipeline}

This phase, basically collects all the OAG\footnote{OAG file (also know as WTF file), is a CSV\cite{csv} file which each row represents a timetable for a Single Flight Number.} from a provider and maps it into routes in JSON\cite{json} format. For each different version of the OAG file, the pipeline creates a new file with all the routes.

%----------------------------------------------------------------
%----------------------------------------------------------------

\section{Functional requirements}

\section{Non functional requirements}

\section{Use cases}

\begin{table}
\begin{tabular}{|>{\raggedright\arraybackslash}p{3.2cm}|>{\raggedright\arraybackslash}p{10cm}|}
\hline
\textbf{Name}                   & Routes offer and demand comparison \textbf{heatmap} \\
\hline
\textbf{ID}                     & UC0 \\
\hline
\textbf{Description}            & Heatmap of the comparison between providers offer and user demand. The heat is represented by the \textit{over requests} of a route. \\
\hline
\textbf{Actors}                 & User \\
\hline
\textbf{Triggers}               & Loading home page \\
\hline
\textbf{Precondition}           & \\
\hline
\textbf{Postcondition}          & Wolrd heatmap with most relevant routes and their heat. \\
\hline
\textbf{Basic Flow}             & \\
\hline
\textbf{Alternate Flow}         & \\
\hline
\textbf{Exceptions}             & \\
\hline
\end{tabular}
\caption{\textit{Routes offer and demand comparison \textbf{heatmap}} use case}
\label{UC0}
\end{table}

\begin{table}
\begin{tabular}{|>{\raggedright\arraybackslash}p{3.2cm}|>{\raggedright\arraybackslash}p{10cm}|}
\hline
\textbf{Name}                   & Offer and demand plot of route \\
\hline
\textbf{ID}                     & UC1 \\
\hline
\textbf{Description}            & Compare the user demand and the providers offer of a specific route from city A to city B in a given date in a plot with two data sets, offer and demand. \\
\hline
\textbf{Actors}                 & User \\
\hline
\textbf{Triggers}               & Request to get comparison of route from city A to city B in a specific date. \\
\hline
\textbf{Precondition}           & City A and city B exists and there is some connection (SFN or Constructed) in the date. \\
\hline
\textbf{Postcondition}          & Plot with the evolution through time of the user demand and air carrier offer. Time limit goes from fist offer apperance to arrival date or current date, depending which comes first. \\
\hline
\textbf{Basic Flow}             & 1. System provides a list of cities under \textit{origin} tag. \\
                                & 2. User selects an origin city. \\
                                & 3. System provides another list of cities. Now with \textit{destination} tag. \\
                                & 4. User selects destination (See exception 1). \\
                                & 5. System provides an interactive calendar. \\
                                & 6. User selects a date of the calendar (See exception 2). \\
                                & 7. System provides the plot of the demand and offer evolution of the route. \\
\hline
\textbf{Alternate Flow}         & Alternate course 1 \\
                                & 1. User \textbf{changes} destination city (See exception 1). \\
                                & 2. Return to basic flow step 6. \\
                                & \\
                                & Alternate course 2 \\
                                & 1. User \textbf{changes} date (See exception 2). \\
                                & 2. Return to basic flow step 7. \\
\hline
\textbf{Exceptions}             & 1. There are no connections between to given cities. \\
                                & 2. There are connections between to given cities, but not in the given date. \\
\hline
\end{tabular}
\caption{\textit{Offer and demand plot of route} use case}
\label{UC1}
\end{table}

\begin{table}
\begin{tabular}{|>{\raggedright\arraybackslash}p{3.2cm}|>{\raggedright\arraybackslash}p{10cm}|}
\hline
\textbf{Name}                   & Offer and demand data set of route \\
\hline
\textbf{ID}                     & UC2 \\
\hline
\textbf{Description}            & Data set of the evolution of the user demand and providers offer in order to create metrics, alerts, etc. \\
\hline
\textbf{Actors}                 & \nameref{mas}, \nameref{dlr} \\
\hline
\textbf{Triggers}               & Request to get data set of route from city A to city B in a specific date. \\
\hline
\textbf{Precondition}           & City A and city B exists and there is some connection (SFN or Constructed) in the date \\
\hline
\textbf{Postcondition}          & Plot with the evolution through time of the user demand and air carrier offer. Time limit goes from fist offer apperance to arrival date or current date, depending which comes first. \\
\hline
\textbf{Basic Flow}             & 1. System provides an HTTP endpoint to request data. \\
                                & 2. The developer does a GET request to the endpoint with an origin, destination and a date (See exception 1). \\
                                & 3. System provides a data set in JSON format with all the demand and offers of the entity. \\
\hline
\textbf{Alternate Flow}         & \\
\hline
\textbf{Exceptions}             & 1. There no connections between city A and city B in the given date. \\
\hline
\end{tabular}
\caption{\textit{Offer and demand data set of route} use case}
\label{UC2}
\end{table}

\begin{table}
\begin{tabular}{|>{\raggedright\arraybackslash}p{3.2cm}|>{\raggedright\arraybackslash}p{10cm}|}
\hline
\textbf{Name}                   & name \\
\hline
\textbf{ID}                     & id \\
\hline
\textbf{Description}            & description \\
\hline
\textbf{Actors}                 & actors \\
\hline
\textbf{Organzational Benefits} & benefits \\
\hline
\textbf{Frequency of Use}       & frequency \\
\hline
\textbf{Triggers}               & trigger \\
\hline
\textbf{Precondition}           & pre \\
\hline
\textbf{Postcondition}          & post \\
\hline
\textbf{Basic Flow}             & main \\
\hline
\textbf{Alternate Flow}         & alt \\
\hline
\textbf{Exceptions}             & exc \\
\hline
\end{tabular}
\caption{\textit{title} use case}
\label{id}
\end{table}

% ...

\begin{table}[H]
\begin{tabular}{>{\raggedleft\arraybackslash}p{2cm}>{\raggedright\arraybackslash}p{12cm}}
\textbf{Name}        & Inception \\
\textbf{Number}      & 1 \\
\textbf{Description} & Regard this an agile project, there has been an inception part where the project was predefined and explained to \company product owners to see if it was viable. \\
\textbf{Duration}    & 29 days \\
\textbf{Start date}  & 02/01/2018 \\
\textbf{ETA}         & 31/01/2018 \\
\end{tabular}
\label{milestone1}
\end{table}

% ...