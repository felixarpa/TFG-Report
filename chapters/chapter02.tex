% Chapter 2: State-of-the-art

\chapter{State-of-the-art}

\label{chapter02}

Since this project is not oriented for \company\ users but the company, the \textit{\nameref{chapter02}} relates to services inside \company. Even so, a brief explanation about other metasearch engines would help to find the gap this project is developed:

\section{Fare aggregators and metasearch engines}

\subsection{Google Flights}

In the last years Google Flights has became the main competitor of \company. The new version is very fast and has a complete new interface, following Android guidelines.
\\\\
Google is one of the top tech companies and has a lot of different platforms. It is a competitor to be aware of, the integration with Gmail, Google Calendar and Android OS makes Google Flight a part of its ecosystem. The traveler may feel comfortable.

\subsection{Kayak}

Kayak has always been the main competitor, both companies started in 2004. Unlike \company, Kayak started with Flights, Hotels and Car hiring. \company\ added those two extra search engines between 2013 and 2014.

\subsection{Expedia}

Launched in November 1998, is one of the oldest fare aggregator and metasearch engine. Apart of its own website, is also a \company\ provider. Some of the prices are taken from Expedia and sometimes the user is redirected to their website to finish their purchase.

\section{\company\ services}

In \company\ the user has never been a product, in fact, one of the statements of \company's culture says \textit{Traveler != Product}\cite{the_road_ahead}.
\\\\
There has never been a project getting value from user information because it does not follows the company culture, so the definition of the problem and the scope of the project must be very accurate to ensure it is fulfilling with \company's strategy\cite{skyscanner_strategy}.

\subsection{Marketplace Engine}

This tribe is formed by five squads, those constantly work to improve the routes and pricing service all along with an efficient search.
\\\\
Marketplace Engine works with data \textit{from the provider to the user}. In other words, it just serves \textbf{information to the user} but does not get any from him/her. All five Squads take all the \textbf{data from providers}.
\\\\
\includegraphics[scale=0.5]{diagrams/state-of-the-art-tribes-mp.png}

\subsection{Date Tribe}

In the other hand, Data Tribe has a lot of squads with services used collect \textbf{data from user activity}. The flow of the information is \textit{from the user to \company}.
\\\\
\includegraphics[scale=0.5]{diagrams/state-of-the-art-tribes-data.png}

\subsection{The gap}

There is no tribe of squad that works with both \textbf{sources of data}: Providers and Users. And here is where the \textit{Heatmap} will be.
\\\\
\includegraphics[scale=0.5]{diagrams/state-of-the-art-tribes-hm.png}

