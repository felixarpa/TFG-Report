% Chapter 4: Stakeholders

\chapter{Stakeholders}

\label{chapter04}

Initially it seemed difficult to find stakeholders and actors in these project apart from the providers. It is not a tool for the user of \company.
\\\\
After talking with the squad Lead and then the Product Own of \squad\ a lot of stakeholders appeared: DeLorean squad, Marketing Automation squad, Data tribe, etc. Each of these stakeholders has different use cases and the project became very interesting for a considerable part of \company.

\section{DeLorean squad} \label{dlr}

DeLorean's Single Flight Number service, also known as \textit{Timetable SFN Service}, provides all the \textbf{current} flights. This is a little bit of a problem when trying to get historical data because Timetable SFN Service does not provide past flights information, it is always \textbf{up-to-date}. In order to get this data it is needed to go one step back in the whole DeLorean data processing: \textit{Timetable Pipeline}.
\\\\
The \thesistitle must look old versions of the file created by the \textit{Timetable Pipeline} to get older routes. Then, \squad\ is interested in the \thesistitle because it will be using Pipeline's data.

\subsection{Product Owner} \label{product_owner}

\textbf{Jen Agerton} is the Product Owner of DeLorean squad. She realized that the \thesistitle is very useful for other squads like \nameref{mas} and providers (air carrier companies).

\subsection{DeLorean's squad Lead}

\textbf{Francisco López} is also the supervisor of this project. We both had the initial idea for this project. He saw an opportunity for the future (after project's delivery) orienting the \thesistitle for Machine Learning purpose: The information that the comparison stores is very useful for constructed routes.

\section{Marketing Automation squad} \label{mas}

Marketing Automation squad enables scalable growth by automating workflows, and the collection of insightful data. They have three main goals:

\begin{itemize}
  \item Provide data to support decision making
  \item Automated, data driven campaign management
  \item Budget process automation
\end{itemize}

The \thesistitle will be very useful for the first goal. The data provided by the Heatmat has high value in marketing decisions. Looking at historical data, \nameref{mas} could post an advertisement about trend routes in a specific time of the year.

\section{Data tribe}

In Data tribe, State Machine squad captures the user actions, so they know where the user gets stuck or if they finally reach the provider of the flight. Other squads like Clan A and Clan B just gets users queries in flights, hotels and car hiring. The second data source of the \thesistitle (user requests and queries) will be obtained from these squads.

\section{Other \company\ developers}

Last but not least, a new service will appear in \company, all developers will be able to use it and build software using the \thesistitle's compared data. For instance, it can be used as a training for a complex Machine Learning\cite{machine_learning_coursera}.
\\\\
The server Application Program Interface, used by the Web UI to visualize all data, will be public inside \company. This and all the documentation will be very helpful for developers.

\section{OAG}

OAG is a company that collects all logistic flights information. \squad reads data from them, it is the main provider of information regarding routes. They are the world's largest network of air travel data to provide accurate, timely, actionable digital information and applications to the world's airlines, airports, government agencies and travel-related service companies\cite{oag}.

\section{Providers}

In the future, providers could take profit from \thesistitle comparison. Companies will be able to know which of their routes or airports work better with user tendencies, they will be able to improve the flights service and make it more efficient, reducing number of flights in \textit{non-profitable} routes. They will also know which are the best places to invest looking at \textit{over-requested} airports.

\section{Traveler}

\company users are one of the main sources of information. Without them, the comparison cannot be made. The results of the comparison can also help them, not directly, but if providers somehow manage flights and routes following the \thesistitle results, the traveler experience will improve.


